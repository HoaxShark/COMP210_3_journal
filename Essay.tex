% Please do not change the document class
\documentclass{scrartcl}

% Please do not change these packages
\usepackage[hidelinks]{hyperref}
\usepackage[none]{hyphenat}
\usepackage{setspace}
\doublespace

% You may add additional packages here
\usepackage{amsmath}

% Please include a clear, concise, and descriptive title
\title{Research methods for alternative controller.}

% Please do not change the subtitle
\subtitle{COMP210 - Research Journal}

% Please put your student number in the author field
\author{1706966}

\begin{document}
	
	\maketitle
	
	\abstract{This work sums up a few HCI research methods. Afterwards going on to select one most relevant for a research study regarding a student made alternative controller and game.}
	
	\section{Introduction}
		This paper will be detailing some research into Human-Computer Interaction research methods. It will begin by talking about a few selected methods and then go into more depth about one of the previously selected methods. The purpose is to provide a solid grasp on how to implement the chosen method successfully, and use the data it collects correctly. All considerations of the possible usability of these methods will be in regard to them being used to test a university students alternative controller and game.
		
	\section{Methods}
		A research method provides a way of collecting data. There are two types of data quantitative and qualitative. "Qualitative research is empirical research where the data are not in the form of numbers"\cite{punch2013introduction} where as "Quantitative research gathers data in a numerical form which can be put into categories, or in rank order, or measured in units of measurement."\cite{simplePsych1}. Methods can provide a mix of these data types or one specifically. When looking to choose a research method it's important to consider these data types and decide what you are looking to gather.

	\subsection{Surveys}
		A survey is a list of questions given to a sample of a particular group of people, with the aim of extracting specific data from them. They are good at quickly and easily collecting data, but if not written well the data they provide can be bad\cite{litwin1995measure}. These can collect both quantitative and qualitative by either asking multiple choice, or open ended questions. A test should be sent out before fully implementing the survey to catch any mistakes. 
		
	\subsection{Task Analysis}
		Task analysis as a method of research, lists set tasks for the user to complete and measures aspects of how they do when trying to complete the task. This requires more interaction from the researcher as they will be present to monitor how the user manages with the tasks\cite{diaper2003handbook}. A lot of consideration needs to go into what the tasks will be and how they are worded. Typically will gather quantitative data.
		
	\subsection{A/B Testing}
		A/B testing takes two versions of the product to be tested. The idea is that users are given the different versions to use, and then data is gathered during their use to see which version most effectively did its job. This type of testing is used widely web page design\cite{azevedo2018b}. With the right tools it is easy to gather huge amounts of data cheaply even for small changes. 
		
	\section{Task Analysis}
		Given the quick break downs of the research methods, task analysis was chosen to be looked into with more depth. As the controller is a physical product that then controls a game, it will be more interesting to see how people with no concept of the product are able to adapt to using it. Setting tasks such as "how long it takes the user to equip the controller" will give an insight into how easy it is to understand without prior knowledge. The paper will now take a look at a few parts of task analysis in greater detail.
		
	\subsection{Decide what the user needs to be able to do}
		To be able to observe the user and gather data it is required that you give them something to do. It is good practice to set the request in a short scenario that guides them to what actions they need to take, and why they are doing it. When coming up with task scenarios to be used in testing, you should first create a list of general goals that users may have when using your product. A question you should stop and think about is: What are the most important tasks that the product should allow the user to complete?\cite{nielsen_norman_tognazzini_2018}
	
	\subsection{Task Scenarios}
		Task scenarios have to provide a goal that makes the user engage with the product as if they were in a natural environment, such as in their home doing this task as normal. Try to avoid requesting users to use a specific feature of the product in the scenario, this way you are able to see how the user chooses to interact with the interface naturally. The tips below are worth thinking about when making task scenarios.\cite{nielsen_norman_tognazzini_2018}
		\begin{itemize}
			\item Have realistic tasks - To do this well consider the target users for the tasks, try to make it something they would do normally. This way they feel more like they are doing the task for real giving you better data results.
			\item Tasks should have an action - Don't ask the user how to do something, ask them to do it! Have scenarios that make the user do their task and not respond verbally.
			\item Don't give clues - It's best to not give the user any leading hints on how to use the interface as this won't give as good results. Don't use the same words that are found on the interface you are trying to test.
		\end{itemize}	
		
		
	\subsection{Cognitive Task Analysis}
	Cognitive task analysis can use various methods, such as interviews or think alouds, to gather additional qualitative data to facilitate a greater depth of insight into why the tasks were performed as they were by the user\cite{clark2007cognitive}. Using these qualitative methods ends up providing less reliable data, and at a greater cost. However with the combined approach you can gather much better data in regards to how to improve the product that it can be very worthwhile\cite{koedinger2016closing}. 
	


	\section{Conclusion}
		To conclude, there are many ways to collect research data and its advantageous to look at the ones available to you and pick what is most suitable for your product. Task analysis is able to provide you useful usability information if you can find the time to observe and the participants to take part. Combining task analysis with another research method such as casual interviews will not only allow you to gather the data from the use of the tasks but may provide insight into ways to improve the product, depending on the chosen test targets.

	
	\bibliographystyle{ieeetran}
	\bibliography{references}
	
\end{document}
